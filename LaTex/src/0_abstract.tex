\section*{Zusammenfassung}

Die vorliegende Arbeit, beschäftigt sich mit der Entwicklung einer Client-Server-Architektur am Beispiel einer Anwendung zur Verwaltung von Selbstbedienungsständen.  
\\
\\
Im ersten Teil dieser Arbeit werden die Grundlagen, die zum späteren Verständnis wichtig sind, erläutert. Es wird erklärt, was ein Selbstbedienungsstand ist, außerdem werden einige gesetzliche Grundlagen vorgestellt. Anschließend wird etwas genauer beleuchtet, was unter einer Client-Server-Architektur zu verstehen ist. Im Anschluss werden unterschiedliche Frameworks dargestellt.
\\
\\
Das zweite Kapitel widmet sich der Anforderungsanalyse, dort werden die wichtigsten Anforderungen genauer betrachtet. Die einzelnen Anforderungen werden kurz erörtert und etwas näher beschrieben.
\\
\\
Darauf aufbauend, wird im dritten Kapitel mit der Konzeption der Anwendung begonnen. Zunächst wird der allgemeine Architekturentwurf präsentiert. Danach folgt im Anschluss der Entwurf des Servers und der Datenbank. Hiernach wird auf die Konzeption der Schnittstellen eingegangen. Der letzte Teil des Kapitels beschäftigt sich mit der Konzeption des Clients.
\\
\\
Im Fokus des vierten Kapitels steht die Realisierung und Implementierung der Anwendung. Zunächst wird die Datenbank implementiert, es werden einige Datenbankabfragen vorgestellt. Nachfolgend wird auf die Implementierung des Servers eingegangen. Näher werden einige Beispielcodeausschnitte erläutert. Zuletzt wird auf die Implementierung des Angular-Clients eingegangen. Dabei wird das Kapitel aufgesplittet in Implementierung der grafische Benutzeroberfläche(Frontend) und Implementierung des Hintergrundbereiches (Backend).
\\
\\
Zum Schluss folgt eine Zusammenfassung und ein Ausblick.


