\section{Fazit und Ausblick}\label{zusammenfassung}

In diesem letzten Kapitel folgt ein kurzes Fazit und es wird ein kurzer Ausblick geben.

\subsection{Fazit}\label{fazit}

Bei der Entwicklung der Anwendung hat sich gezeigt, dass eine gute und ausführliche Planung der Anwendung wichtig ist. Vornherein muss genau überlegt werden, was für Funktionen die Anwendung beinhalten soll. Eine genaue Auflistung ist hier vom Vorteil.
\\
Danach muss überlegt werden, welche Daten brauche ich und wie speicher ich diese am besten in der Datenbank. Dabei ist es wichtig, gründlich und genau zu arbeiten, ich habe die Erfahrung machen müssen, dass es im Nachhinein schwierig ist, noch weitere Tabellenspalten hinzuzufügen.
\\
\\
Im Anschluss folgt die Konzeption des Servers, hierbei ist es wichtig, die Pakete und Klassen möglichst geordnet anzulegen. Außerdem sollte darauf geachtet werden, das bekannte Architekturmuster umgesetzt werden. Diese haben sich häufig in der Praxis bewährt grade bei größeren Projekten helfen diese dabei, eine Struktur beizubehalten.
\\
\\
Als besonders wichtig hat sich die genaue Planung der Schnittstellen herausgestellt. Es sollte eine Schnittstellenbeschreibung angefertigt werden. Dieses erleichtert die Entwicklung des Clients ungemein. Da dort übersichtlich alle Schnittstellen dargestellt werden können.
\\
\\
Es hat sich herausgestellt, dass sich die Entwicklung eines Angular-Clients vereinfachen lässt, in dem vorab eine Skizze der Benutzeroberfläche angefertigt wird. Diese lässt sich in einzelne Bestandteile zerlegt und somit gut in Komponenten aufteilen.
\\
\\
Abschließend lässt sich sagen, dass es wichtig ist, die Software gut zu planen, dadurch lassen sich viele Fehler vermeiden, die sonst erst bei der Realisierung der Software auffallen.
Es hat sich gezeigt, dass sich grade kleine Fehler, die am Anfang des Projektes gemacht werden, am Ende des Projektes zu immer größeren Fehlern führen.

\newpage

\subsection{Ausblick}\label{ausblick}

Die Anwendung kann und soll noch weiter entwickelt werden, dieses ist leicht möglich, da diese modular aufgebaut worden ist. Folgende Funktionen sind noch denkbar:

\begin{itemize}
	\item Erstellung eines Warenverkaufsschildes im PDF Format
	\item Erstellung unterschiedlicher Statistiken (Jahresstatistik, Monatsstatistik)
	\item Einbindung eines Kassenzählprotokolls
	\item Berechnung des Verlustes, durch Diebstahl usw.
	\item App-Entwicklung 
\end{itemize}

Außerdem soll ein Praxistest erfolgen, mit Besitzern von Selbstbedienungsständen, dabei soll sich herausstellen, ob diese Anwendung in der Paxis erfolgversprechend eingesetzt werden kann.
\\
