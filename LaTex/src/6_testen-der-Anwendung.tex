\section{Testen der Anwendung}\label{testen_der_Anwendung}

Zum Abschluss folgt eine Tabelle, aus dieser ist zu entnehmen, welche der definierten Anforderungen aus Kapitel \ref{anforderungsanalyse}
erfüllt oder teilweise erfüllt sind. 
\\
\begin{table}[htbp]
\begin{tabular}{|c|c|c|c|c|c|}
	\hline
		Nr. & Anforderung & Erfüllt &Teilweise &Nicht  & Uner-\\
		 &  &  &erfüllt & erfüllt & füllbar\\
		\hline
		1 & Benutzer und Selbstbedienungsstand neu anlegen & X & & &  \\
		\hline
		2 & Benutzername oder Passwort ändern &  & X* &  &\\
		\hline
		3 & Benutzer löschen &  & X*  & & \\
		\hline
		4 & Mehrbenutzer &  & X*  &  &\\
		\hline	
		5 & Selbstbedienungsstand anlegen &  & X* & & \\
		\hline
		6 & Selbstbedienungsstand, Name ändern &  & X*  &  &\\
		\hline
		7 & Selbstbedienungsstand löschen &  & X* & & \\
		\hline
		8 & Selbstbedienungsstand, Benutzer hinzufügen &  & X* & & \\
		\hline
		9 & Selbstbedienungsstand, neuen Benutzer erstellen &  & X*  & & \\
		\hline
		10 & Ware anlegen & X &  & & \\
		\hline
		11 &  Ware löschen & X &  & & \\
		\hline
		12 &  Warendaten ändern & X &  & & \\
		\hline
		13 & Warenbewegung eingeben & X &  &  &\\
		\hline
		14 & Tagesabschluss erstellen & X &  &  &\\
		\hline
		15 & JSON Web Token  & X &  & & \\
		\hline
		16 &  Verschlüsseltes Passwort in Datenbank speichern & X &  & & \\
		\hline
		17 & Erweiterbares System & X &  & & \\
		\hline
		18 &  Plattformunabhängig & X &  &  & \\
		\hline
		19 &  Standortunabhängig &  & X & & \\
		\hline
		20 & Responsive-Webdesign &  & X &  & \\
		\hline
		21 &  Tagesabschluss als PDF-Dokument & X &  & & \\
		\hline
		22 & Leichte Bedienbarkeit &  & X & & \\
		\hline
		23 &  Übersichtliches Design &  & X &  &\\
		\hline
		24 &  Deutsche Textausgabe & X  &  & & \\
		\hline
	\end{tabular}
	X* Funktion ist bereits auf dem Server implementiert.
	\caption{Überprüfung der Anforderungen}
	\label{tab:zu2}
\end{table}

Die Tabelle \ref{tab:zu2} listet nochmals die Anforderungen auf. In einigen Spalten X* zu sehen (Zeilen 2 bis 9), diese beutetet das die Funktion bereits auf dem Server implementiert worden sind. 
\\ Nach Fertigstellung der Funktion auf dem Server wurden diese getestet. Getestet wurde mit dem Programm Postman.\footnote{https://www.postman.com}. Damit ist es möglich unterschiedliche HTTP-Anfragen durchzuführen.
\\
\\
\newpage
Die Anforderung Nr. 19 kann nue Teilweise erfüllt werden, da für die nutzung der Anwendung ein Internetzugang erforderlich ist. Abhilfe könnte eine App schaffen, die die Daten lokal speichert und sobald Internetempfang vorhanden ist diese überträgt.
\\
\\
Die Anforderung Nr. 20 Responsive-Webdesign wurde nur teilweise erfüllt, die Navigationsleiste erinnert an klassische Desktopanwendungen. Ein Hamburger-Menü ist hingegen besser bedienbar von mobilen Endgeräten. 
\\
\\
Die Anforderungen 22 und 23 lassen sich schwer überprüfen, da dieses jeder Benutzer anderes wahrnimmt. Dieses lässt sich am besten in einem Praxistest mit unterschiedlichen Anwendern überprüfen. 