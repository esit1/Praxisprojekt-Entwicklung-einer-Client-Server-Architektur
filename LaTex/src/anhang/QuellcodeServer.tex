\subsection *{Quellcode Server (Ausschnitt)}\label{QuellcodeServer}


\lstset{language=java}
\begin{lstlisting}[frame=tb, caption={Das Listing zeigt die Klasse EndOfTheDay im Paket model}, label={lst:EndOfTheDay}]
import lombok.Getter;
import lombok.Setter;

import javax.persistence.*;

/**
* Class represents the table "tb_end of_the_day".
*/
@Table(name = "tb_end of_the_day", indexes = {
	@Index(name = "cl_user_nr", columnList = "cl_user_nr"),
	@Index(name = "cl_goods_nr", columnList = "cl_goods_nr"),
	@Index(name = "cl_self_service_stand_nr", 
	columnList = "cl_self_service_stand_nr")
})
@Setter
@Getter
@Entity
public class EndOfTheDay {
	
	/**
	* ID End of the Day.
	*/
	@EmbeddedId
	private EndOfTheDayId id;
	
	/**
	* Refers to a user ID.
	*/
	@ManyToOne
	@JoinColumn(name = "cl_user_nr")
	private User userNr;
	
	/**
	* Refers to a self-Service-Stand name.
	*/
	@ManyToOne
	@JoinColumn(name = "cl_self_service_stand_nr")
	private SelfServiceStand selfServiceStandName;
	
	/**
	* The number of goods counted.
	*/
	@Column(name = "cl_graduation_count")
	private Integer graduationCount;
	
	/**
	* The number of computationally determined goods sold.
	*/
	@Column(name = "cl_graduation_sold")
	private Integer graduationSold;
	
	/**
	* The number of mathematically determined income in euros.
	*/
	@Column(name = "cl_graduation_revenue")
	private Double graduationRevenue;
}
\end{lstlisting}

%##############################################################################################

\lstset{language=java}
\begin{lstlisting}[frame=tb, caption={Das Listing zeigt die Klasse EndOfTheDayId im Paket model}, label={lst:EndOfTheDayId}]
/**
* Class represents the table ID "tb_end of_the_day".
*/
@Getter
@Setter
@Embeddable
public class EndOfTheDayId implements Serializable {
	
	/**
	* serialVersionUID
	*/
	private static final long serialVersionUID = 4548732131340178144L;
	
	/**
	* End of the day date.
	*/
	@Column(name = "cl_graduation_date", nullable = false)
	private LocalDate graduationDate;
	
	/**
	* The commodity number.
	*/
	@Column(name = "cl_goods_nr", nullable = false)
	private Integer goodsNr;
	
	/**
	* Overwritten hash code method.
	*
	* @return Objects
	*/
	@Override
	public int hashCode() {
		return Objects.hash(goodsNr, graduationDate);
	}
	
	/**
	* Overwritten equals method.
	*
	* @param o goodsNr and graduationDate
	* @return Objects
	*/
	@Override
	public boolean equals(Object o) {
		if (this == o) return true;
		if (o == null || Hibernate.getClass(this) != Hibernate.getClass(o)) return false;
		EndOfTheDayId entity = (EndOfTheDayId) o;
		return Objects.equals(this.goodsNr, entity.goodsNr) &&
		Objects.equals(this.graduationDate, entity.graduationDate);
	}
}
\end{lstlisting}

%##############################################################################################

\lstset{language=java}
\begin{lstlisting}[frame=tb, caption={Das Listing zeigt das Interface ReceiptGoodRepository im Paket repository}, label={lst:ReceiptGoodRepository}]
/**
* This class contains the database queries of the table tb_receipt_goods.
*/
@Repository
public interface ReceiptGoodRepository extends JpaRepository<ReceiptGoods, Integer> {
	
	/**
	* Query, finds a record with the incoming ID.
	* @param receiptGoodId The receipt goods Id.
	* @return record receiptGoods
	*/
	ReceiptGoods findByReceiptGoodId(Integer receiptGoodId);
	
	/**
	* Query, provides a list of all self-service booths that have the ID entered.
	* @param selfServiceStandNId The self-service-stand Id.
	* @return list receiptGoods
	*/
	List<ReceiptGoods> findBySelfServiceStandNr_SelfServiceStandNIdOrderByDateReceiptAsc(Integer selfServiceStandNId);
	
	/**
	* Query, provides a list of all self-service booths that have the ID and product name entered.
	* @param goodsName The goods name.
	* @param selfServiceStandNId The self-service-stand Id.
	* @return list receiptGoods
	*/
	List<ReceiptGoods> findByGoodsNr_GoodsNameNr_GoodsNameAndSelfServiceStandNr_SelfServiceStandNId(String goodsName, Integer selfServiceStandNId);
	
	/**
	* Query, finds a list of all dates and goods In that match.
	* @param dateReceipt The receipt date.
	* @param goodsId The goods ID.
	* @return list receiptGoods
	*/
	List<ReceiptGoods> findByDateReceiptAndGoodsNr_GoodsId(LocalDate dateReceipt, Integer goodsId);
}
\end{lstlisting}
%##############################################################################################

\lstset{language=java}
\begin{lstlisting}[frame=tb, caption={Das Listing zeigt die Klasse UnitController im Paket controller}, label={lst:UnitController}]
/**
* The role Controller class outputs a list of all existing units.
* It is possible to create new units.
* Units can also be changed and deleted, but only if they are not yet used in another table.
*
* <p>
* Provides the following endpoints: <br>
* - GET /role/ <br>
* - POST /role/{unitName} <br>
* - DELETE /role/{oldUnitName} <br>
* - PUT /role/{oldUnitName}/{newUnitName} <br>
*/
@RestController
@RequestMapping("/unit")
public class UnitController {
	
	/**
	* Logger of the unitController class.
	*/
	private static final Logger LOGGER = LoggerFactory.getLogger(UnitController.class);
	
	/**
	* unitRepository to handle unit information.
	*/
	private final UnitRepository unitRepository;
	
	/**
	* goodsRepository to handle goods information.
	*/
	private final GoodsRepository goodsRepository;
	
	public UnitController(UnitRepository unitRepository, GoodsRepository goodsRepository) {
		this.unitRepository = unitRepository;
		this.goodsRepository = goodsRepository;
	}
	
	/**
	* Method returns a sorted list with all units.
	*
	* @return json list with all units
	*/
	@GetMapping(path = "/", produces = MediaType.APPLICATION_JSON_VALUE)
	private ResponseEntity<Object> getUnits() {
		
		//Query sorted list, sorted by unit name
		return new ResponseEntity<>(unitRepository.findByOrderByUnitNameAsc(), HttpStatus.OK);
	}
	
	/**
	* Method creates a new unit, but only if it does not yet exist.
	*
	* @param unitName Name of the unit
	* @return Http Status code: Create if a new unit has been created, otherwise no content
	* and created unit.
	*/
	@PostMapping(path = "/{unitName}", produces = MediaType.APPLICATION_JSON_VALUE)
	private ResponseEntity<Unit> createUnit(@PathVariable String unitName) {
		
		//excess whitespace is removed
		unitName = unitName.trim();
		
		Unit newUnit = null;
		
		//It is checked whether the unit already exists.
		if (!unitRepository.existsByUnitNameIs(unitName)) {
			
			//save the new unit
			newUnit = unitRepository.save(new Unit(unitName));
			
			LOGGER.info("The unit " + unitName + " has been created.");
		} else {
			LOGGER.info("The unit " + unitName + " is already there.");
		}
		
		// Set the http status by unit; CREATED if it is null, NO_CONTENT otherwise
		final HttpStatus httpStatus = null != newUnit
		? HttpStatus.CREATED
		: HttpStatus.NO_CONTENT;
		
		return new ResponseEntity<>(newUnit, httpStatus);
	}
	
	/**
	* The method only deletes a unit from the list if it has not yet been used.
	*
	* @param unitName unit name
	* @return message in Json format and Http status code
	*/
	@DeleteMapping(path = "/{unitName}", produces = MediaType.APPLICATION_JSON_VALUE)
	private ResponseEntity<JSONObject> deleteUnit(@PathVariable String unitName) {
		
		//create new json object
		JSONObject message = new JSONObject();
		
		//excess whitespace is removed
		unitName = unitName.trim();
		
		//It is checked whether the unit is not used in the "goods" table.
		if (!goodsRepository.existsByGoodsUnitNr_UnitName(unitName)) {
			
			Unit deleteUnit = new Unit();
			deleteUnit.setUnitId(unitRepository.findByUnitName(unitName).getUnitId());
			
			//Delete Unit at the database
			unitRepository.delete(deleteUnit);
			
			LOGGER.info("The unit " + unitName + " has been deleted.");
			message.put("Message", "The unit " + unitName + " has been deleted.");
			return new ResponseEntity<>(message, HttpStatus.OK);
		} else {
			LOGGER.info("The unit " + unitName + " is already in use. Deletion is no longer possible.");
			message.put("Message", "The unit " + unitName + " is already in use. Deletion is no longer possible.");
			return new ResponseEntity<>(message, HttpStatus.CONFLICT);
		}
	}
	
	/**
	* The method only changes the name of a unit if it has not yet been used in the Goods table.
	*
	* @param oldUnitName old unit name
	* @param newUnitName new unit name
	* @return message in Json format and Http status code
	*/
	@PutMapping(path = "/{oldUnitName}/{newUnitName}", produces = MediaType.APPLICATION_JSON_VALUE)
	private ResponseEntity<JSONObject> updateUnit(@PathVariable String oldUnitName, @PathVariable String newUnitName) {
		
		//create new json object
		JSONObject message = new JSONObject();
		
		//excess whitespace is removed
		oldUnitName = oldUnitName.trim();
		newUnitName = newUnitName.trim();
		
		//It is checked whether the unit is not used in the "goods" table.
		if (!goodsRepository.existsByGoodsUnitNr_UnitName(oldUnitName)) {
			
			Unit changeUnit = new Unit();
			changeUnit.setUnitId(unitRepository.findByUnitName(oldUnitName).getUnitId());
			
			changeUnit.setUnitName(newUnitName);
			
			//change unit at the database
			unitRepository.save(changeUnit);
			
			LOGGER.info("The name of the unit has been changed from " + oldUnitName + " to " + newUnitName + ".");
			message.put("Message", "The name of the unit has been changed from " + oldUnitName + " to " + newUnitName + ".");
			return new ResponseEntity<>(message, HttpStatus.OK);
		} else {
			LOGGER.info("The unit " + oldUnitName + " is already in use. Change is no longer possible.");
			message.put("Message", "The unit " + oldUnitName + " is already in use. Change is no longer possible.");
			return new ResponseEntity<>(message, HttpStatus.CONFLICT);
		}
	}
}
\end{lstlisting}













\subsection *{Quellcode Client (Ausschnitt)}\label{QuellcodeClient}

\lstset{language=java}
\begin{lstlisting}[frame=tb, caption={Das Listing zeigt die CSS-Datei main-screen.component }, label={lst:main-screen.component}]
.container {
	display: grid;
	grid-template-columns: 0.5fr 2fr 8fr 2fr 0.5fr;
	grid-template-rows: 0.4fr 3.4fr 0.4fr 0.4fr;
	gap: 0px 0px;
	grid-auto-flow: row;
	grid-template-areas:
	"navbar navbar navbar navbar navbar"
	"main main main main main"
	"selectSelf selectSelf . . ."
	"footer footer footer footer footer";
}

.navbar {
	grid-area: navbar;
	background: var(--colorGray);
}

.main {
	grid-area: main;
	background: white;
}

.footer {
	grid-area: footer;
	background: var(--colorGray);
}

.selectSelf {
	grid-area: selectSelf;
	background: white;
}

html,
body {
	height: 100%;
	background: radial-gradient(var(--colorBackground2), var(--colorBackground1));
	overflow: hidden;
}

.container-main-screen {
	background: white;
}	
\end{lstlisting}
%##############################################################################################
\lstset{language=java}
\begin{lstlisting}[frame=tb, caption={Das Listing zeigt die CSS-Datei main-screen.component }, label={lst:main-screen.component}]
const myRoutes: Routes = [
{
	path: '',
	component: MainScreenComponent,
	children: [
	{
		path: 'goods',
		component: GoodsComponent,
		children: [
		{
			path: 'overview',
			component: OverviewGoodsComponent
		},
		{
			path: 'new',
			component: NewGoodsComponent
		},
		{
			path: 'change',
			component: ChangeGoodsComponent
		}, {
			path: 'settings',
			component: SettingsGoodsComponent
		}
		]
	},
	{
		path: 'receipt',
		component: ReceiptComponent,
		children: [
		{
			path: 'new',
			component: NewReceiptComponent
		},
		{
			path: 'overview',
			component: OverviewReceiptComponent
		},
		]
	},
	{
		path: 'user',
		component: UserComponent,
		children: [
		{
			path: 'change',
			component: ChangeUserComponent
		},
		{
			path: 'create',
			component: CreateUserComponent
		},
		{
			path: 'overview',
			component: OverviewUserComponent
		}
		]
	},
	{
		path: 'statistics',
		component: StatisticsComponent,
		children: [
		{
			path: 'month',
			component: MonthComponent
		},
		{
			path: 'year',
			component: YearComponent
		}
		]
	},
	{
		path: 'self',
		component: SelfServiceStandComponent,
		children: [
		{
			path: 'add',
			component: AddUserSelfComponent
		},
		{
			path: 'change',
			component: ChangeSelfComponent
		},
		{
			path: 'overview',
			component: OverviewSelfComponent
		}
		]
	},
	{
		path: 'end',
		component: EndOfDayComponent,
		children: [
		{
			path: 'overview',
			component: OverviewEndComponent
		},
		{
			path: 'new',
			component: NewEndComponent
		},
		{
			path: 'create',
			component: EndOfDayPdfComponent
		}
		]
	},
	{
		path: '',
		component: MainMenuComponent
	},
	{
		path:'**',
		component: PageNotFoundComponent,
	},
	]
}
]

@NgModule({
	declarations: [],
	imports: [
	CommonModule,
	RouterModule.forChild((myRoutes))
	],
	exports: [RouterModule]
})
export class MainScreenRoutesModule {
}
\end{lstlisting}
%##############################################################################################

\lstset{language=java}
\begin{lstlisting}[frame=tb, caption={Das Listing zeigt die TypeScript-Datei change-goods.ts }, label={lst:main-screen.component2}]

/**
* Class provides an overview of all goods.
*/
export class OverviewGoodsComponent implements OnInit {
	
	//empty list goods
	goodsList: Goods[] = [];
	
	//Heading of the table
	header: string[] = ['Warenname', 'Einheit', 'Preis', 'Bemerkung', 'im Verkauf'];
	
	//Saves whether a result was found
	searchResult: boolean = true;
	
	//List of search results
	searchResultList: Goods[] = [];
	
	
	constructor(private client: HttpClient,
	private communication: ServerCommunicationServiceService
	) {
	}
	
	ngOnInit(): void {
		this.loadGoods();
		this.searchResultList = this.goodsList;
	}
	
	//Filter the list of goods according to the search results and save them in the list
	search(event: Event) {
		
		//Old search results are deleted
		this.searchResultList = [];
		
		//Saves the keystrokes
		const filterValue = (event.target as HTMLInputElement).value;
		
		//Searches the list for the word
		for (let key in this.goodsList) {
			if (this.goodsList.hasOwnProperty(key)) {
				if (
				this.goodsList[key].goodsName.includes(filterValue) ||
				this.goodsList[key].unitName.includes(filterValue) ||
				this.goodsList[key].goodsPrice.toString().includes(filterValue) ||
				this.goodsList[key].goodsNote.includes(filterValue)
				) {
					//Saves all found experiences in the list
					this.searchResultList.push((this.goodsList[key]));
				}
			}
			//If not found, the variable is set to false.
			this.searchResult = this.searchResultList.length > 0;
		}
	}
	
	// Loads a list of goods from the server
	private loadGoods() {
		
		//set url
		let url = 'goods/' + localStorage.getItem('key_selfServiceStand_name');
		
		//communication Server
		this.communication
		.serverCommunication(url, '', 'get', true)
		.subscribe((response: any) => {
			
			//Saves the goods in a list
			for (var entry of response) {
				this.goodsList.push({
					goodsId: entry.goodsId,
					goodsName: entry.goodsNameNr.goodsName,
					unitName: entry.goodsUnitNr.unitName,
					goodsPrice: entry.goodsPrice,
					goodsNote: entry.goodsNote,
					goodsActive: entry.goodsActive
				});
			}
		}, error => {
			console.log(error);
		}
		);
	}
}
\end{lstlisting}
%##############################################################################################