\subsection *{Entwurf der Schnittstellen}\label{Schnittstellen}

\textbf{Benutzer anlegen}

\textit{Client-Anfrage}
\\
Authentifizierung: JWT\\
Befehl: POST user/new\\
Daten: Benutzername, Passwort\\

\textit{Server-Antwort}
\\
Daten:\\
Erfolg: Statuscode 201\\
oder bei Misserfolg: Statuscode 400, Fehlermeldung: Dem Benutzer wurde bereits einen  Selbstbedienungsstand mit dem Namen XY zugewiesen.\\
oder bei einer ungültigen Authentifizierung: Statuscode 401 

\noindent\rule{\textwidth}{1pt}
\\
\textbf{Schnittstelle Tagesabschluss(EndOfTheDayController)}
\\
Der EndOfTheDayController stellt die folgenden Schnittstellen zur Verfügung:

\begin{itemize}
	\itemsep0pt
	\item  POST /endOfDay/
	\item  PUT /endOfDay/
	\item  GET /endOfDay/
\end{itemize}

Diese Schnittstellen dienen der Verwaltung des Tagesabschlusses, für jede Ware kann pro Tag nur ein Abschluss erstellt werden. Außerdem darf nur ein Benutzer mit Administratorrechten dieses ausführen.
Die Schnittstelle POST /endOfDay/ sorgt dafür das ein neuer Eintrag erstellt werden kann.
Mithilfe der Schnittstelle PUT /endOfDay/ kann dieser Eintrag editiert werden.
Die Schnittstelle GET /endOfDay/ übermittelt eine Auflistung, darin enthalten sind unter anderem die Angaben, der theoretisch verkauften Waren und theoretische Einnahmen (ohne Diebstahl, beschädigte Ware usw.). 
\\
\\
\textit{\underline{POST /endOfDay/}}
\\
\\
\NumTabs{4}				
Eingabeparameter: \tab				- LocalDate graduationDate (Datum des Tagesabschlusses)\\
\tab\tab								- String codesName (Name der Ware)		\\
\tab\tab								- String userName (Name des Benutzers)\\
\tab\tab								- String selfServiceStandName (Name des \\
\tab\tab 								Selbstbedienungsstandes)\\
\tab\tab								- int graduationCount (Die Anzahl der vorhanden Waren, \\
\tab \tab 								ermitteln durch zählen, wiegen o. Ä. )\\									
\\
Rückgabewerte: \tab 					Erfolg: HttpStatus.OK und Meldung.\\
\tab \tab 									Misserfolg: HttpStatus.NO CONTENT und Meldung\\
\\
Beispielrückgabelmeldung:	
\\
\\
\textit{\underline{PUT /endOfDay/}}
\\
\\
\NumTabs{4}				
Eingabeparameter: \tab				- Optional, LocalDate graduationDate (geändertes Datum \\
\tab\tab 							des Tagesabschlusses)\\
\tab\tab							- Optional,	String codesName (geänderter Name der Ware)		\\
\tab\tab							- Optional,	String userName (geänderter Name des Benutzers)\\
\tab\tab							- Optional,	int graduationCount (geänderte Anzahl der \\
\tab\tab vorhanden Waren)\\
\tab\tab											- LocalDate oldDate (Datum des alten Tagesabschlusses, \\
\tab\tab vor der Änderung)\\
\tab\tab												- String oldGoodsName (Name der alten Ware, \\
\tab\tab vor der Änderung)\\
\\
Rückgabewerte: \tab 					Erfolg: HttpStatus.OK und Meldung.\\
\tab \tab 									Misserfolg: HttpStatus.NO CONTENT und Meldung\\
\\
Beispielrückgabelmeldung:	
\\
\\
\\

%%%%%%%%%%%%%%%%%%%%%%%%%%%%%%%%%%%%%%%%%%%%%%%%%%%%%%%%%%%%%%%%%%%%%%%%%%%%%%%%%%%%%%%%%%%%%%%%%%%%%%%%%%%%%%%%%%%%%%%%%%%%%%%%%%%%%%%%%%%%%%%%%%%%%%%%%
\textbf{Schnittstelle Waren(GoodsController)}
\\
Der EndOfTheDayController stellt die folgenden Schnittstellen zur Verfügung:

\begin{itemize}
	\itemsep0pt
	\item  POST /goods/
	\item  PUT /goods/
	\item  DELETE /goods/
	\item  GET /goods/
	\item  GET /goods/{selfServiceStandName}/{userName}	
	\item  GET /goods//{selfServiceStandName}/{userName}/{goodsName}
\end{itemize}

Mithilfe dieser Schnittstellen lässt sich neue Ware anlegen (POST /goods/), verändern (PUT /goods/) oder auch löschen (DELETE /goods/). Diese Schnittstellen können nur von jemanden mit Administratorrechten genutzt werden. Eine Ware kann nur gelöscht werden, wenn diese nicht genutzt wird. Die beiden Schnittstellen GET /goods/ geben eine Übersicht der Waren zurück, dieses kann eine Liste oder eine einzelne Ware sein.
\\
\\
\textit{\underline{POST /endOfDay/}}
\\
\\
\NumTabs{4}				
Eingabeparameter: \tab			- String goodsName(Warenname)\\
\tab \tab                        		- String unitName	(Wareneinheit)\\
\tab \tab                         		- int goodsPrice (Preis der Ware)\\
\tab \tab                         		- String goodsNote (Bemerkung)    \\                  
\tab \tab                         		- String selfServiceStandName \\
\tab \tab								 (Name des Selbstbedienungsstandes)\\
\tab \tab                        		- String userName (Benutzername)\\
\\
Rückgabewerte: \tab 					Erfolg: HttpStatus.OK und Meldung.\\
\tab \tab 								Misserfolg: HttpStatus.NO CONTENT und Meldung\\
\\
Beispielrückgabelmeldung:	
\\
\\
\textit{\underline{PUT /endOfDay/}}
\\
\\
\NumTabs{4}				
Eingabeparameter: \tab			- String goodsName( Warenname)\\
\tab \tab                        		- optional String unitName	(geänderte Wareneinheit)\\
\tab \tab                         -	optional	int goodsPrice (geänderter Preis der Ware)\\
\tab \tab                         		- optional String goodsNote (geänderte Bemerkung)    \\                  
\tab \tab                         		- String selfServiceStandName \\
\tab \tab								(Name des Selbstbedienungsstandes)\\
\tab \tab                        		- String userName (Benutzername)\\
\tab \tab                        		- optional String changeGoodsName (Der neue Name, der Ware)\\
\\
Rückgabewerte: \tab 					Erfolg: HttpStatus.OK und Meldung.\\
\tab \tab 								Misserfolg: HttpStatus.NO CONTENT und Meldung\\
\\
Beispielrückgabelmeldung:	
\\
\\	
\textit{\underline{POST /endOfDay/}}
\\
\\
\NumTabs{4}				
Eingabeparameter: \tab			- String goodsName(Warenname)\\
\tab \tab                        		- String unitName	(Wareneinheit)\\
\tab \tab                         		- int goodsPrice (Preis der Ware)\\
\tab \tab                         		- String goodsNote (Bemerkung)    \\                  
\tab \tab                         		- String selfServiceStandName \\
\tab \tab								 (Name des Selbstbedienungsstandes)\\
\tab \tab                        		- String userName (Benutzername)\\
\\
Rückgabewerte: \tab 					Erfolg: HttpStatus.OK und Meldung.\\
\tab \tab 								Misserfolg: HttpStatus.NO CONTENT und Meldung\\
\\
Beispielrückgabelmeldung:	
\\
\\
\textit{\underline{DELETE /endOfDay/}}
\\
\\
\NumTabs{4}				
Eingabeparameter: \tab			- String goodsName( Warenname)\\                
\tab \tab                         		- String selfServiceStandName \\
\tab \tab								(Name des Selbstbedienungsstandes)\\
\tab \tab                        		- String userName (Benutzername)\\
\\
Rückgabewerte: \tab 					Erfolg: HttpStatus.OK und Meldung.\\
\tab \tab 								Misserfolg: HttpStatus.NO CONTENT und Meldung\\
\\
Beispielrückgabelmeldung:	
\\
\\
\textit{\underline{GET /goods//{selfServiceStandName}/{userName}}}
\\
\\
Rückgabewerte: \tab 					Erfolg: HttpStatus.OK und Meldung.\\
\\
Beispielrückgabelmeldung:	
\\
\\
\textit{\underline{GET /goods//{selfServiceStandName}/{userName}}}
\\
\\
Rückgabewerte: \tab 					Erfolg: HttpStatus.OK und Meldung.\\
\\
Beispielrückgabelmeldung:	
\\
\\
%%%%%%%%%%%%%%%%%%%%%%%%%%%%%%%%%%%%%%%%%%%%%%%%%%%%%%%%%%%%%%%%%%%%%%%%%%%%%%%%%%%%%%%%%%%%%%%%%%%%%%%%%%%%%%%%%%%%%%%%%%%%%%%%%%%%%%%%%%%%%%%%%%%%%%%%%

\textbf{Schnittstelle Warenname(GoodsNameController)}
\\
Der GoodsNameController stellt die folgenden Schnittstellen zur Verfügung:

\begin{itemize}
	\itemsep0pt
	\item  POST /goodsName/{goodsName}
	\item  PUT /goodsName/
	\item  DELETE /goodsName/
	\item  GET /goodsName/

\end{itemize}

Die Schnittstellen der Klasse GoodsNameController enthalten Schnittstellen die dafür verantwortlich sind Warennamen hinzuzufügen (POST /goodsName/), ändern (PUT /goodsName/) und löschen (DELETE /goodsName/). Außerdem kann eine Liste mit allen verfügbaren Namen ausgegeben werden.
Jeder Warenname kann nur einmalig vergeben werden. \\
Der Warennamen wie in einer separaten Tabelle gespeichert, dieses hat den Vorteil das Redundanzen vermieden werden. Außerdem geht so verhindert das ist unterschiedliche Schreibweisen für ein und dieselbe Ware existieren.
\\
\\
\textit{\underline{GET /goodsName/}}
\\
\\
\NumTabs{4}				
Eingabeparameter: \tab					- keine Variablen erforderlich
\\
Rückgabewerte: \tab 					Erfolg: Json Liste mit allen Warenamen.\\
\\
Beispielrückgabelmeldung:	
\\
\\
%§§§§§§§§§§§§§§§§§§§§§§§§§§§§§§§§§§§§§§§§§§§§§§§§§§
\\
\\
\textit{\underline{POST /goodsName/{goodsName}}}
\\
\\
Rückgabewerte: \tab 					Erfolg: HttpStatus Created\\
\tab \tab 								Misserfolg: HttpStatus No CONTENT\\
\\
Beispielrückgabelmeldung:	
\\
\\
%§§§§§§§§§§§§§§§§§§§§§§§§§§§§§§§§§§§§§§§§§§§§§§§§§§
\\
\\
\textit{\underline{DELETE /goodsName/{goodsName}}}
\\
\\
Rückgabewerte: \tab 					Erfolg: HttpStatus.OK\\
\tab \tab 								Misserfolg: HttpStatus Conflict\\
\\
Beispielrückgabelmeldung:	
\\
\\
%§§§§§§§§§§§§§§§§§§§§§§§§§§§§§§§§§§§§§§§§§§§§§§§§§§
\\
\\
\textit{\underline{DELETE /goodsName/{oldGoodsName}/{newGoodsName}}}
\\
\\
Rückgabewerte: \tab 					Erfolg: HttpStatus OK\\
\tab \tab 								Misserfolg: HttpStatus Conflict\\
\\
Beispielrückgabelmeldung:	
\\
\\
%§§§§§§§§§§§§§§§§§§§§§§§§§§§§§§§§§§§§§§§§§§§§§§§§§§
%%%%%%%%%%%%%%%%%%%%%%%%%%%%%%%%%%%%%%%%%%%%%%%%%%%%%%%%%%%%%%%%%%%%%%%%%%%%%%%%%%%%%%%%%%%%%%%%%%%%%%%%%%%%%%%%%%%%%%%%%%%%%%%%%%%%%%%%%%%%%%%%%%%%%%%%%
\textbf{Schnittstelle Wareneingang(EndOfTheDayController)}
\\
Der EndOfTheDayController stellt die folgenden Schnittstellen zur Verfügung:

\begin{itemize}
	\itemsep0pt
	\item  POST /receipt/
	\item  PUT /receipt/
	\item  DELETE /receipt/{receiptGoodsNr}
	\item  GET /receipt/{receiptGoodsNr}
	\item  GET /receipt/{selfServiceStandName}
	\item  GET /receipt/{selfServiceStandName}/{goodsName}
\end{itemize}

Mithilfe dieser Schnittstellen wird der Wareneingang und Warenausgang protokolliert. Mithilfe der Schnittstelle POST /receipt/ wird ein neuer Wareneingang oder Warenausgang hinzugefügt. PUT /receipt/ ändert diesen und DELETE /receipt/{receiptGoodsNr} löscht diesen Eintrag.
Die Schnittstelle GET /receipt/{receiptGoodsNr} übergibt einen einzelnen Datensatz. Eine Liste mit allen Eingängen und Ausgängen, einer bestimmten Ware überliefert die Schnittstelle GET /receipt/{selfServiceStandName}/{goodsName}. Die Schnittstelle GET /receipt/{selfServiceStandName} liefert ebenfalls eine Liste, allerdings mit allen Ein-und Ausgängen.  Bei den letzten beiden Schnittstellen muss der Selbstbedienungsstandnamen angeben werden, damit ein eindeutiger Zuordnung stattfinden kann. 
\\
\\
\textit{\underline{POST /receipt/}}
\\
\\
\NumTabs{4}				
Eingabeparameter: \tab			- LocalDate receiptDate(Datum Bestandsveränderung)\\
\tab \tab                        		- int goodsPieces\\
\tab \tab 								(Anzahl der Ware, die hinzugefügt oder entnommen wurde) \\
\tab \tab                         		- String goodsName \\
\tab \tab                         		- String userName \\                  
\tab \tab                         		- String selfServiceStandName\\
\\
Rückgabewerte: \tab 					Erfolg: HttpStatus OK und Meldung.\\
\tab \tab 								Misserfolg: HttpStatus NO CONTENT und Meldung\\
\\
Beispielrückgabelmeldung:	
\\
\\
%§§§§§§§§§§§§§§§§§§§§§§§§§§§§§§§§§§§§§§§§§§§§§§§§§§
\\
\\
\textit{\underline{PUT /receipt/}}
\\
\\
\NumTabs{4}				
Eingabeparameter: \tab			- int receiptID (ID)\\
\tab \tab                        		- optional LocalDate receiptDate(Datum Bestandsveränderung)\\\\
\tab \tab                        		- optional int goodsPieces\\
\tab \tab 								(Anzahl der Ware, die hinzugefügt oder entnommen wurde) \\
\tab \tab                         		- optional String goodsName \\
\tab \tab                         		- optional String userName \\                  
\\
Rückgabewerte: \tab 					Erfolg: HttpStatus OK \\
\tab \tab 								Misserfolg: HttpStatus NO CONTENT \\
\\
Beispielrückgabelmeldung:	
\\
\\
%§§§§§§§§§§§§§§§§§§§§§§§§§§§§§§§§§§§§§§§§§§§§§§§§§§
\\
\\
\textit{\underline{DELETE /receipt/{receiptGoodsNr}}}
\\              
\\
Rückgabewerte: \tab 					Erfolg: HttpStatus OK \\
\tab \tab 								Misserfolg: HttpStatus Conflict \\
\\
Beispielrückgabelmeldung:	
\\
\\
%§§§§§§§§§§§§§§§§§§§§§§§§§§§§§§§§§§§§§§§§§§§§§§§§§§
\\
\\
\textit{\underline{GET /receipt/{receiptGoodsNr}}}
\\              
\\
Rückgabewerte: \tab 					Erfolg: ein Datensatz \\
\\
Beispielrückgabelmeldung:	
\\
\\
%§§§§§§§§§§§§§§§§§§§§§§§§§§§§§§§§§§§§§§§§§§§§§§§§§§
\\
\\
\textit{\underline{GET /receipt/{selfServiceStandName}}}
\\                
\\
Rückgabewerte: \tab 					Erfolg: Liste mit allen Datensätzen, die dem Selbstbedingungsstand zugeordnet sind\\
\\
Beispielrückgabelmeldung:	
\\
\\
%§§§§§§§§§§§§§§§§§§§§§§§§§§§§§§§§§§§§§§§§§§§§§§§§§§
\\
\\
\textit{\underline{GET /receipt/{selfServiceStandName}}}
\\                
\\
Rückgabewerte: \tab 					Erfolg: Liste mit allen Datensätzen, die dem Selbstbedingungsstand zugeordnet sind und die dem Warennamen entsprechen\\
\\
Beispielrückgabelmeldung:	
\\
\\
%%%%%%%%%%%%%%%%%%%%%%%%%%%%%%%%%%%%%%%%%%%%%%%%%%%%%%%%%%%%%%%%%%%%%%%%%%%%%%%%%%%%%%%%%%%%%%%%%%%%%%%%%%%%%%%%%%%%%%%%%%%%%%%%%%%%%%%%%%%%%%%%%%%%%%%%%
\textbf{Schnittstelle Benutzerrollen(RoleController)}
\\
Der RoleController stellt die folgenden Schnittstellen zur Verfügung:

\begin{itemize}
	\itemsep0pt
	\item  GET /role/
	\item PUT /role//userRole/
\end{itemize}

Diese Schnittstelle GET /role/ gibt eine sortierte Liste mit allen verfügbaren Benutzerrollen zurück.
PUT /role//userRole/ bewirkt das eine Benuterrolle eines Benutzer geändert wird. Nur ein Benutzer mit Administratorrechten darf Benutzerrollen ändern.
\\
\\
\textit{\underline{GET /role/}}
\\
\\
Rückgabewerte: \tab 					Erfolg: HttpStatus OK und Liste mit allen Benutzerrollen.\\
\\
Beispielrückgabelmeldung:	
\\
\\
%§§§§§§§§§§§§§§§§§§§§§§§§§§§§§§§§§§§§§§§§§§§§§§§§§§
\\
\\
\textit{\underline{PUT /role//userRole/}}
\\
\\
\NumTabs{4}				
Eingabeparameter: \tab			- String changeUserName (Benutzer dessen Rolle geändert wird)\\
\tab \tab                        		- String userName (Name des Benutzer, der die Benutzerrolle ändert)\\\\
\tab \tab                        		- String selfServiceStandName\\
\tab \tab                         		- optional String goodsName \\
\tab \tab                         		- String newRole (Name der neuen Role) \\                  
\\
Rückgabewerte: \tab 					Erfolg: HttpStatus OK \\
\tab \tab 								Misserfolg: HttpStatus NO CONTENT \\
\\
Beispielrückgabelmeldung:	
\\
\\
%%%%%%%%%%%%%%%%%%%%%%%%%%%%%%%%%%%%%%%%%%%%%%%%%%%%%%%%%%%%%%%%%%%%%%%%%%%%%%%%%%%%%%%%%%%%%%%
\textbf{Schnittstelle Selbstbedingungsstand (SelfServiceStandController)}
\\
Der SelfServiceStandController stellt die folgenden Schnittstellen zur Verfügung:

\begin{itemize}
	\itemsep0pt
	\item  POST /selfServiceStand/{userName}/{selfServiceStandName}
	\item  POST /selfServiceStand/addUser/
	\item  DELETE /selfServiceStand/{removeUserName}/{userName}/{selfServiceStandName}
	\item GET /selfServiceStand/{selfServiceStandName}
\end{itemize}


Diese Schnittstellen helfen bei der Verwaltung eines Selbstbedienungsstandes. Ein neuer Selbstbedienungsstand wird mit der Schnittstelle POST /selfServiceStand/{userName}/{selfServiceStandName} erstellt, der Benutzer der diesen erstellt hat, erhält Administratorrechte.\\ Die Schnittstelle POST /selfServiceStand/addUser/ fügt einen bereits existierenden Benutzer, einen Selbstbedienungsstand hinzu, dabei kann die Benutzerrolle frei gewählt werden. Sobald ein Benutzer entfernt werden soll, hilft die Schnittstelle DELETE /selfServiceStand/{removeUserName}/{userName}/{selfServiceStandName}.
Eine Liste mit allen Benutzern die dem Selbstbedienungsstand zugeordnet sind, wird mithilfe der Schnittstelle GET /selfServiceStand/{selfServiceStandName} ausgeben.
\\
\\
\textit{\underline{POST /selfServiceStand/{userName}/{selfServiceStandName}}}
\\
\\
Rückgabewerte: \tab 					Erfolg: HttpStatus OK und Meldung.\\
\tab \tab 								Misserfolg: HttpStatus Conflict und Meldung\\
\\
Beispielrückgabelmeldung:	
\\
\\
%§§§§§§§§§§§§§§§§§§§§§§§§§§§§§§§§§§§§§§§§§§§§§§§§§§
\\
\\
\textit{\underline{POST /selfServiceStand/addUser/}}
\\
\\
\NumTabs{4}				
Eingabeparameter: \tab			- String addingUserName (Benutzer der hinzugefügt wird)\\
\tab \tab                        		- String userName (Name des Benutzer, der den neuen Benutzer hinzufügt)\\\\
\tab \tab                        		- String selfServiceStandName\\                 
\\
Rückgabewerte: \tab 					Erfolg: HttpStatus OK \\
\tab \tab 								Misserfolg: HttpStatus Conflict \\
\\
Beispielrückgabelmeldung:	
\\
\\
%%%%%%%%%%%%%%%%%%%%%%%%%%%%%%%%%%%%%%%%%%%%%%%%%%%%%%%%%%%%%%%%%%%%%%%%%%%%%%%%%%%%%%%%%%%%%%%
\textit{\underline{DELETE /selfServiceStand/{removeUserName}/{userName}/{selfServiceStandName}}}
\\
\\
Rückgabewerte: \tab 					Erfolg: HttpStatus OK und Meldung.\\
\tab \tab 								Misserfolg: HttpStatus Conflict und Meldung\\
\\
Beispielrückgabelmeldung:	
\\
\\
%§§§§§§§§§§§§§§§§§§§§§§§§§§§§§§§§§§§§§§§§§§§§§§§§§§
%%%%%%%%%%%%%%%%%%%%%%%%%%%%%%%%%%%%%%%%%%%%%%%%%%%%%%%%%%%%%%%%%%%%%%%%%%%%%%%%%%%%%%%%%%%%%%%%%%%%%%%%%%%%%%%%%%%%%%%%%%%%%%%%%%%%%%%%%%%%%%%%%%%%%%%%%
\textbf{Schnittstelle Einheiten(UnitController)}
\\
Der UnitController stellt die folgenden Schnittstellen zur Verfügung:

\begin{itemize}
	\itemsep0pt
	\item  POST /unit/{unitName}
	\item  PUT /unit/{oldUnitName}/{newUnitName}
	\item  DELETE /unit//{unitName}
	\item  GET /unit//{unitName}
\end{itemize}
Einheiten wie z.B. Stück, KG, Sack usw. können mit der Schnittstelle POST /unit/ erstellt werden. Verändert werden können diese mit der Schnittstelle PUT /unit/. Soll eine Einheit gelöscht werden ist dieses mit der Schnittstelle DELETE /unit//{unitName} möglich. Allerdings können Änderung und Löschungen nur durchgeführt werden, wenn die Einheit noch nicht verwendet wird.
Eine Liste der verfügbaren Einheiten gibt die Schnittstelle GET /unit//{unitName} aus.
\\
\\
\textit{\underline{POST /unit/{unitName}}}
\\
\\
Rückgabewerte: \tab 					Erfolg: HttpStatus Create und Meldung.\\
\tab \tab 								Misserfolg: HttpStatus No Content\\
\\
Beispielrückgabelmeldung:	
\\
\\
%§§§§§§§§§§§§§§§§§§§§§§§§§§§§§§§§§§§§§§§§§§§§§§§§§§
\textit{\underline{PUT /unit/{oldUnitName}/{newUnitName}}}
\\
\\
Rückgabewerte: \tab 					Erfolg: HttpStatus OK und Meldung.\\
\tab \tab 								Misserfolg: HttpStatus Conflict\\
\\
Beispielrückgabelmeldung:	
\\
\\
%§§§§§§§§§§§§§§§§§§§§§§§§§§§§§§§§§§§§§§§§§§§§§§§§§§
\textit{\underline{DELETE /unit//{unitName}}}
\\
\\
Rückgabewerte: \tab 					Erfolg: HttpStatus OK und Meldung.\\
\tab \tab 								Misserfolg: HttpStatus Conflict\\
\\
Beispielrückgabelmeldung:	
\\
\\
%§§§§§§§§§§§§§§§§§§§§§§§§§§§§§§§§§§§§§§§§§§§§§§§§§§
\textit{\underline{GET /unit//{unitName}}}
\\
\\
Rückgabewerte: \tab 					Erfolg: HttpStatus OK und Liste mit allen Einheiten.\\
\\
Beispielrückgabelmeldung:	
\\
\\
%§§§§§§§§§§§§§§§§§§§§§§§§§§§§§§§§§§§§§§§§§§§§§§§§§§
%%%%%%%%%%%%%%%%%%%%%%%%%%%%%%%%%%%%%%%%%%%%%%%%%%%%%%%%%%%%%%%%%%%%%%%%%%%%%%%%%%%%%%%%%%%%%%%%%%%%%%%%%%%%%%%%%%%%%%%%%%%%%%%%%%%%%%%%%%%%%%%%%%%%%%%%%
\textbf{Schnittstelle Wareneingang(EndOfTheDayController)}
\\
Der EndOfTheDayController stellt die folgenden Schnittstellen zur Verfügung:

\begin{itemize}
	\itemsep0pt
	\item  POST /user/newAdmin
	\item  POST /user/newUser
	\item  PUT /user/
	\item  GET /user/{username}
\end{itemize}

Die Benutzerverwaltung kann mithilfe nachfolgender Schnittstellen durchgeführt werden. Ein neuer Benutzer und ein neuer Selbstbedienungsstand werden mit der Schnittstelle POST /user/newAdmin erstellt. Der neu erstellte Benutzer, wird automatisch dem erstellten Selbstbedienungsstand als Administrator zugeordnet. Der Name des Selbstbedienungsstandes und des Benutzers dürfen noch nicht vergeben sein.
\\
Ein Administrator eines  Selbstbedienungsstandes kann einen neuen Benutzer zu diesem hinzufügen, der Benutzer wird dabei neu erstellt, die Benutzerrolle kann frei gewählt werden, dieses geschieht mit der Schnittstelle POST /user/newUser. 
\\
Sobald ein neuer Benutzer erstellt wird wie zuerst überprüft ob der Benutzername bereits vergeben ist, sollte dieses der Fall sein, kann der Benutzer nicht erstellt werden. Die Schnittstelle PUT /user/ ist zum Editieren des Benutzers gedacht. Ein einzelner Benutzer kann mit der Schnittstelle GET /user/{username} ausgegeben werden.
\\
\\
\textit{\underline{POST /user/newAdmin}}
\\
\\
\NumTabs{4}				
Eingabeparameter: \tab			- String userName (Benutzername)\\
\tab \tab                        		- String userPassword (Passwort)\\
\tab \tab                         		- String selfServiceStandName (Name des Selbstbedienungsstandes, der neu erstellt wird) \\
\tab \tab                         		- String email (E-Mail Adresse)\\                  
\\
Rückgabewerte: \tab 					Erfolg: HttpStatus Createt und Meldung.\\
\tab \tab 								Misserfolg: HttpStatus NO CONTENT und Meldung\\
\\
Beispielrückgabelmeldung:	
\\
\\
%§§§§§§§§§§§§§§§§§§§§§§§§§§§§§§§§§§§§§§§§§§§§§§§§§§
\textit{\underline{POST /user/newUser}}
\\
\\
\NumTabs{4}				
Eingabeparameter: \tab			- String userName (Benutzername)\\
\tab \tab                        		- String userPassword (Passwort)\\
\tab \tab                         		- String selfServiceStandName (Name des Selbstbedienungsstandes, zu dem der Benutzer hinzugefügt wird. ) \\
\tab \tab                         		- String email (E-Mail Adresse) \\                  
\\
Rückgabewerte: \tab 					Erfolg: HttpStatus Createt und Meldung.\\
\tab \tab 								Misserfolg: HttpStatus NO CONTENT und Meldung\\
\\
Beispielrückgabelmeldung:	
\\
\\
%§§§§§§§§§§§§§§§§§§§§§§§§§§§§§§§§§§§§§§§§§§§§§§§§§§
\textit{\underline{PUT /user/}}
\\
\\
\NumTabs{4}				
Eingabeparameter: \tab			- optional String userName (Benutzername)\\
\tab \tab                        		- optional String userPassword (Passwort)\\
\tab \tab                         		- optional String selfServiceStandName (Name des Selbstbedienungsstandes, zu dem der Benutzer hinzugefügt wird. ) \\
\tab \tab                         		- optional String email (E-Mail Adresse)\\
\tab \tab                         		- optional String changeUserName (neuer Benutzername)\\                   
\\
Rückgabewerte: \tab 					Erfolg: HttpStatus Ok und Meldung.\\
\tab \tab 								Misserfolg: HttpStatus NO Conflict und Meldung\\
\\
Beispielrückgabelmeldung:	
\\
\\
%§§§§§§§§§§§§§§§§§§§§§§§§§§§§§§§§§§§§§§§§§§§§§§§§§§
\textit{\underline{GET /user/{username}}}
\\                  
\\
Rückgabewerte: \tab 					Erfolg: HttpStatus Createt und Benutzerdetails\\
\\
Beispielrückgabelmeldung:	
\\
\\
%§§§§§§§§§§§§§§§§§§§§§§§§§§§§§§§§§§§§§§§§§§§§§§§§§§